\documentclass[12pt]{article}
\usepackage{amsmath}
\usepackage{mathtools}
\usepackage{amssymb}
\usepackage{hyperref}
\hypersetup{
    colorlinks=true,
    linkcolor=blue,
    filecolor=magenta,      
    urlcolor=cyan,
}
\usepackage{physics}
\usepackage[margin=1.2in]{geometry}

\usepackage[utf8]{inputenc}

\begin{document}
\begin{center}
\textbf{\Large Week 7 Report} \\
\end{center}

I've been spending the past few days rereading \href{https://arxiv.org/abs/1307.0471}{Quantum Support Vector Machine}. The goal is to fully understand the algorithm so that I can implement it on IBM QX. Quantum Algorithm for SVM is a very complex one and hard to understand because it is built up on so many other quantum algorithms. The paper, \href{https://arxiv.org/abs/1307.0471}{Quantum Support Vector Machine}, just merely describes the general ideas for the algorithm and doesn't describe in details how to build implement complex operations with quantum gates. \\

In the past few weeks, I have been looking at QSVM implementation on IBM QX. I found out that IBM QX has its own built-in function for QSVM, so users can directly call the built-in function from python and use it for SVM classification problem. But one thing to keep in mind is the dataset format for the input data. The built-in QSVM function only takes in classical dataset with certain format, more detail can be found \href{https://github.com/Qiskit/qiskit-acqua-tutorials/blob/master/artificial_intelligence/datasets.py}{here}. So, if we want to use other datasets other than the default datasets, we need to write a function to reformat it. I have played around with built-in function with the default datasets. Attached in emails are some python notebooks that I have played around. \\

I have attempted to look at the code for quantum algorithms for SVM in QISKIT-acqua github. Basically, we call the built-in python function for QSVM, it calls the following \href{https://github.com/Qiskit/qiskit-acqua/blob/master/qiskit_acqua/algomethods.py}{run algorithm function}. After many calls to many other files, algorithm.run will end up at the \href{https://github.com/Qiskit/qiskit-acqua/blob/master/qiskit_acqua/classicalsvm/svm_rbf_kernel.py}{Quantum Radial basis function kernel} if we use RBF kernel for SVM, or it will end up at \href{https://github.com/Qiskit/qiskit-acqua/blob/master/qiskit_acqua/svm/svm_qkernel.py}{Quantum enhanced kernel}. After reading the code at the mentioned file, I find out that, the built-in python function for QSVM use quantum algorithm to implement the kernel matrix, which can be found \href{https://github.com/Qiskit/qiskit-acqua/blob/master/qiskit_acqua/svm/quantum_circuit_kernel.py}{here}. Other than that, the built-in function use cvxopt to find solution for the quadratic programming problem to solve for parameters $\alpha$ and b. Hence, this built-in function doesn't utilize all the quantum algorithms mentioned in the paper. \\

Meanwhile, \href{https://github.com/JinlongHuang/quantum-SVM/blob/master/quantum_circuit.ipynb}{the following implementation of QSVM on QuTip} involves a lot of complex gates, e.g. SWAP gate, SNOT gate, ..., and those gates again can be approximated by IBM Q gates. But the circuit would still be very complex, and it is not clear to me yet how to implement some certain gates using basic gates provided by IBM Q. \\

My quest to implement SVM circuit on IBM QX is stuck right now, because to implement the full algorithm on IBM QX is a complex problem that I am not yet able to fully comprehend. Meanwhile,  I found the following papers, \href{https://arxiv.org/pdf/1703.10793.pdf}{Implementing a distance - based classifier with a quantum interference circuit} and \href{http://file.scirp.org/pdf/JQIS_2018030614075233.pdf}{Empirial Analysis of a Quantum Classifier implemented on a IBM's 5Q Quantum Computer}. Those papers attempt to simulate similar problems to QSVM. So I decide to have a look at those for the upcoming weeks. 

\end{document}